% Options for packages loaded elsewhere
\PassOptionsToPackage{unicode}{hyperref}
\PassOptionsToPackage{hyphens}{url}
\PassOptionsToPackage{dvipsnames,svgnames,x11names}{xcolor}
%
\documentclass[
  10pt,
]{krantz}
\usepackage{amsmath,amssymb}
\usepackage{lmodern}
\usepackage{setspace}
\usepackage{iftex}
\ifPDFTeX
  \usepackage[T1]{fontenc}
  \usepackage[utf8]{inputenc}
  \usepackage{textcomp} % provide euro and other symbols
\else % if luatex or xetex
  \usepackage{unicode-math}
  \defaultfontfeatures{Scale=MatchLowercase}
  \defaultfontfeatures[\rmfamily]{Ligatures=TeX,Scale=1}
\fi
% Use upquote if available, for straight quotes in verbatim environments
\IfFileExists{upquote.sty}{\usepackage{upquote}}{}
\IfFileExists{microtype.sty}{% use microtype if available
  \usepackage[]{microtype}
  \UseMicrotypeSet[protrusion]{basicmath} % disable protrusion for tt fonts
}{}
\makeatletter
\@ifundefined{KOMAClassName}{% if non-KOMA class
  \IfFileExists{parskip.sty}{%
    \usepackage{parskip}
  }{% else
    \setlength{\parindent}{0pt}
    \setlength{\parskip}{6pt plus 2pt minus 1pt}}
}{% if KOMA class
  \KOMAoptions{parskip=half}}
\makeatother
\usepackage{xcolor}
\IfFileExists{xurl.sty}{\usepackage{xurl}}{} % add URL line breaks if available
\IfFileExists{bookmark.sty}{\usepackage{bookmark}}{\usepackage{hyperref}}
\hypersetup{
  pdftitle={Arts escèniques},
  pdfauthor={Ramiro Palau},
  colorlinks=true,
  linkcolor={Maroon},
  filecolor={Maroon},
  citecolor={Blue},
  urlcolor={Blue},
  pdfcreator={LaTeX via pandoc}}
\urlstyle{same} % disable monospaced font for URLs
\usepackage[left=3cm,right=3cm,top=0.85cm,bottom=3cm]{geometry}
\usepackage{color}
\usepackage{fancyvrb}
\newcommand{\VerbBar}{|}
\newcommand{\VERB}{\Verb[commandchars=\\\{\}]}
\DefineVerbatimEnvironment{Highlighting}{Verbatim}{commandchars=\\\{\}}
% Add ',fontsize=\small' for more characters per line
\usepackage{framed}
\definecolor{shadecolor}{RGB}{248,248,248}
\newenvironment{Shaded}{\begin{snugshade}}{\end{snugshade}}
\newcommand{\AlertTok}[1]{\textcolor[rgb]{0.94,0.16,0.16}{#1}}
\newcommand{\AnnotationTok}[1]{\textcolor[rgb]{0.56,0.35,0.01}{\textbf{\textit{#1}}}}
\newcommand{\AttributeTok}[1]{\textcolor[rgb]{0.77,0.63,0.00}{#1}}
\newcommand{\BaseNTok}[1]{\textcolor[rgb]{0.00,0.00,0.81}{#1}}
\newcommand{\BuiltInTok}[1]{#1}
\newcommand{\CharTok}[1]{\textcolor[rgb]{0.31,0.60,0.02}{#1}}
\newcommand{\CommentTok}[1]{\textcolor[rgb]{0.56,0.35,0.01}{\textit{#1}}}
\newcommand{\CommentVarTok}[1]{\textcolor[rgb]{0.56,0.35,0.01}{\textbf{\textit{#1}}}}
\newcommand{\ConstantTok}[1]{\textcolor[rgb]{0.00,0.00,0.00}{#1}}
\newcommand{\ControlFlowTok}[1]{\textcolor[rgb]{0.13,0.29,0.53}{\textbf{#1}}}
\newcommand{\DataTypeTok}[1]{\textcolor[rgb]{0.13,0.29,0.53}{#1}}
\newcommand{\DecValTok}[1]{\textcolor[rgb]{0.00,0.00,0.81}{#1}}
\newcommand{\DocumentationTok}[1]{\textcolor[rgb]{0.56,0.35,0.01}{\textbf{\textit{#1}}}}
\newcommand{\ErrorTok}[1]{\textcolor[rgb]{0.64,0.00,0.00}{\textbf{#1}}}
\newcommand{\ExtensionTok}[1]{#1}
\newcommand{\FloatTok}[1]{\textcolor[rgb]{0.00,0.00,0.81}{#1}}
\newcommand{\FunctionTok}[1]{\textcolor[rgb]{0.00,0.00,0.00}{#1}}
\newcommand{\ImportTok}[1]{#1}
\newcommand{\InformationTok}[1]{\textcolor[rgb]{0.56,0.35,0.01}{\textbf{\textit{#1}}}}
\newcommand{\KeywordTok}[1]{\textcolor[rgb]{0.13,0.29,0.53}{\textbf{#1}}}
\newcommand{\NormalTok}[1]{#1}
\newcommand{\OperatorTok}[1]{\textcolor[rgb]{0.81,0.36,0.00}{\textbf{#1}}}
\newcommand{\OtherTok}[1]{\textcolor[rgb]{0.56,0.35,0.01}{#1}}
\newcommand{\PreprocessorTok}[1]{\textcolor[rgb]{0.56,0.35,0.01}{\textit{#1}}}
\newcommand{\RegionMarkerTok}[1]{#1}
\newcommand{\SpecialCharTok}[1]{\textcolor[rgb]{0.00,0.00,0.00}{#1}}
\newcommand{\SpecialStringTok}[1]{\textcolor[rgb]{0.31,0.60,0.02}{#1}}
\newcommand{\StringTok}[1]{\textcolor[rgb]{0.31,0.60,0.02}{#1}}
\newcommand{\VariableTok}[1]{\textcolor[rgb]{0.00,0.00,0.00}{#1}}
\newcommand{\VerbatimStringTok}[1]{\textcolor[rgb]{0.31,0.60,0.02}{#1}}
\newcommand{\WarningTok}[1]{\textcolor[rgb]{0.56,0.35,0.01}{\textbf{\textit{#1}}}}
\usepackage{longtable,booktabs,array}
\usepackage{calc} % for calculating minipage widths
% Correct order of tables after \paragraph or \subparagraph
\usepackage{etoolbox}
\makeatletter
\patchcmd\longtable{\par}{\if@noskipsec\mbox{}\fi\par}{}{}
\makeatother
% Allow footnotes in longtable head/foot
\IfFileExists{footnotehyper.sty}{\usepackage{footnotehyper}}{\usepackage{footnote}}
\makesavenoteenv{longtable}
\usepackage{graphicx}
\makeatletter
\def\maxwidth{\ifdim\Gin@nat@width>\linewidth\linewidth\else\Gin@nat@width\fi}
\def\maxheight{\ifdim\Gin@nat@height>\textheight\textheight\else\Gin@nat@height\fi}
\makeatother
% Scale images if necessary, so that they will not overflow the page
% margins by default, and it is still possible to overwrite the defaults
% using explicit options in \includegraphics[width, height, ...]{}
\setkeys{Gin}{width=\maxwidth,height=\maxheight,keepaspectratio}
% Set default figure placement to htbp
\makeatletter
\def\fps@figure{htbp}
\makeatother
% Make links footnotes instead of hotlinks:
\DeclareRobustCommand{\href}[2]{#2\footnote{\url{#1}}}
\setlength{\emergencystretch}{3em} % prevent overfull lines
\providecommand{\tightlist}{%
  \setlength{\itemsep}{0pt}\setlength{\parskip}{0pt}}
\setcounter{secnumdepth}{5}
\usepackage{booktabs}
\usepackage[catalan]{babel}
\usepackage{longtable}

%no se  ue fan
%\usepackage[bf,singlelinecheck=off]{caption}
%\usepackage{Alegreya}
%\usepackage[scale=.7]{sourcecodepro}
%\usepackage{framed,color}

\usepackage[skins]{tcolorbox}

\definecolor{colortip}{RGB}{81,183,73}
\definecolor{colornote}{RGB}{119,136,153}
\definecolor{colorwarn}{RGB}{255,83,59}
\definecolor{colorinfo}{RGB}{92, 122, 234}
\definecolor{colorcuidao}{RGB}{251,188,5}

\tcbset{
  colbacktitle=white,
  enhanced,
  boxrule=9pt,
  sharp corners,
  attach boxed title to top left={yshift=-1.5mm, xshift=5mm},
  colback=white, 
  coltext=black, 
  leftrule=.3mm,
  rightrule=.3mm,
  bottomrule=.3mm,
  toprule=.3mm,
  boxsep=1pt, 
  arc=4pt 
}
\newtcolorbox{rmdcuidao}[1]{
  title=#1,
  drop small lifted shadow,
  coltitle=colorcuidao,
  colframe=colorcuidao, 
  colback=colorcuidao!5!
}
\newtcolorbox{rmdtip}[1]{
  title=#1,
  drop small lifted shadow,
  coltitle=colortip,
  colframe=colortip, 
  colback=colortip!5!
}

\newtcolorbox{rmdnote}[1]{
  title=#1,
  drop small lifted shadow,
  coltitle=colornote,
  colframe=colornote,
  colback=colornote!5!
}

\newtcolorbox{rmdwarn}[1]{
  title=#1,
  drop small lifted shadow,
  coltitle=colorwarn,
  colframe=colorwarn,
  colback=colorwarn!5!
}

\newtcolorbox{rmdinfo}{
  drop large lifted shadow,
  coltitle=colorinfo,
  colframe=colorinfo,
  colback=colorinfo!5!,
  capture=hbox
}

%\makeatother
% marca d'aigua
% https://github.com/callegar/LaTeX-draftwatermark
% https://github.com/callegar/LaTeX-everypage
% https://ctan.org/pkg/draftwatermark
% draftwatermark espacio inestable everypage
%\usepackage[angle=270,text=Espacio\ inestable,color=gray,pos={0.5in,3.5in},scale=0.25]{draftwatermark}
% draftwatermark 和 everypage 由同一个人维护,后者将停止维护
% https://github.com/CTeX-org/ctex-kit/issues/331
%\RecustomVerbatimEnvironment{Highlighting}{Verbatim}{commandchars=\\\{\}}
\ifLuaTeX
  \usepackage{selnolig}  % disable illegal ligatures
\fi
\usepackage[]{natbib}
\bibliographystyle{apalike}

\title{Arts escèniques}
\usepackage{etoolbox}
\makeatletter
\providecommand{\subtitle}[1]{% add subtitle to \maketitle
  \apptocmd{\@title}{\par {\large #1 \par}}{}{}
}
\makeatother
\subtitle{Projecte de pràctiques d'empresa del grau\\
F.P. ASISX\\
I.E. Maria Enriquez\\
\emph{Gandia 2022}}
\author{Ramiro Palau}
\date{2022-04-20}

\begin{document}
\maketitle

{
\hypersetup{linkcolor=}
\setcounter{tocdepth}{2}
\tableofcontents
}
\setstretch{1}
\hypertarget{about}{%
\chapter*{About}\label{about}}


Documentació dels projectes proposats a realitzar en el desenvolupament de les pràctiques realitzades en \textbf{Espacio Inestable}.

\hypertarget{usage}{%
\section*{Usage}\label{usage}}


Anirem detallant els diferents projectes que se suggereixen per a digitalitzar \emph{les arts escèniques}, amb una breu descripció de la seua posada en funcionament, requisits, pros i contres.

En cas afirmatiu s'aniria desenvolupant la manera d'implantar-lo, cost, i detallarem el seu funcionament i configuració.

La idea, es que aquest document siga una guia per mantenir els projectes elaborats o replicar-los en altres llocs, i anar actualitzant els recursos, com forme evolucione la tecnologia i necessitats de l'empresa.

\hypertarget{streaming}{%
\chapter{Streaming}\label{streaming}}

\emph{Implantació d'un sistema de streaming per a la sala}

Es vol posar un sistema de streaming per a la sala i les seues funcions, o per tindre una còpia de les obres per poder-les editar més tard.

\hypertarget{materials}{%
\section{Materials}\label{materials}}

\emph{Es requeriria}

\begin{enumerate}
\def\labelenumi{\arabic{enumi}.}
\item
  \textbf{Servidor}, que devia ser un ordinador de sobretaula, que es miraran els requisits mínims segons el servei que finalment es donen a terme.
\item
  \textbf{Un servidor de disc}, que podria ser el servidor dels altres apartats per guardar una còpia de les diferents càmeres. Si hi ha pressupost, es faria un \href{https://ca.wikipedia.org/wiki/RAID}{RAID}.
\item
  \textbf{Càmeres ip} i millors \textbf{micròfons} que envien el so per la connexió de xarxa de la càmera.
\item
  \href{https://www.vadavo.com/blog/switch-poe-que-es-y-que-tipos-hay/}{\textbf{Switch}} \href{https://ca.wikipedia.org/wiki/Power_over_Ethernet}{\textbf{PoE}} , es podria posar una de 4 ports en l'escenari, per a felicitar l'escalat posterior del projecte que no són massa cars, i un switch (aquest no PoE) en les oficines per centralitzar la xarxa. i tirar cable entre ells i les càmeres. Reduint la distància del PoE a les càmeres ip.
\item
  \textbf{Conte} en una plataforma de streaming.
\item
  \textbf{Programa de streaming} (OBS, Livestream Studio, Wirecast, vMix, mimoLive o Tricaster) alternatives gratuïtes.
\item
  Un ordinador o portàtil que duga a terme les tasques de \textbf{producció}, podria ser un dels ordinadors de l'empresa que gasta per a disseny, s'utilitzaria en el moment de fer la producció en viu. Per a fer streaming d'una obra editada a posteriori, de les còpies fetes en el servidor de disc, no faria falta.
\end{enumerate}

\hypertarget{projecte-dinstillaciuxf3}{%
\section{Projecte d'instil·lació}\label{projecte-dinstillaciuxf3}}

Es faria fent tres xarxes diferents, una per al circuit d'imatge i so de les càmeres, altra per al wifi de la sala i la tercera per als equips d'administració. Per això, necessitem un switch en l'oficina que puga fer diverses xarxes virtuals, per tindre-les separades. (no volem que un espectador puga entrar en els ordinadors de l'oficina, o sature internet i done latència al streaming)

\hypertarget{pressupost}{%
\section{Pressupost}\label{pressupost}}

\hypertarget{software}{%
\section{Software}\label{software}}

Programes de streaming, OBS, Livestream Studio, Wirecast, mimoLive.

Es recomana per a començar OBS o Livestream, en cas que l'ordinador que farà de centre de producció siga un Mac, l'opció serà OBS.

\begin{longtable}[]{@{}
  >{\raggedright\arraybackslash}p{(\columnwidth - 2\tabcolsep) * \real{0.2917}}
  >{\raggedright\arraybackslash}p{(\columnwidth - 2\tabcolsep) * \real{0.7083}}@{}}
\caption{Programes}\tabularnewline
\toprule
\begin{minipage}[b]{\linewidth}\raggedright
\textbf{Programes}
\end{minipage} & \begin{minipage}[b]{\linewidth}\raggedright
\textbf{Desc}
\end{minipage} \\
\midrule
\endfirsthead
\toprule
\begin{minipage}[b]{\linewidth}\raggedright
\textbf{Programes}
\end{minipage} & \begin{minipage}[b]{\linewidth}\raggedright
\textbf{Desc}
\end{minipage} \\
\midrule
\endhead
\href{https://obsproject.com/}{OBS} & Programari gratis per a gravar video i streaming. \\
\href{https://streamlabs.com/}{Livestream Studio} & Per a Windows \\
\href{https://www.telestream.net/}{Wirecast} & De pagament. \\
\href{https://mimolive.com/}{MimoLive} & Pagament processional. \\
\bottomrule
\end{longtable}

\hypertarget{hardware-necessari}{%
\section{Hardware necessari}\label{hardware-necessari}}

\hypertarget{servidor}{%
\subsection{Servidor}\label{servidor}}

Es comenta que es pot reutilitzar l'ordinador que es te en venda d'entrades, es pot mirar a veure si compleix els requisits mínims requerits. En principi, no crec que faça falta un massa potent per al que volem posar.

En el servidor s'implementaran serveis de DNS, Servidor de fulles web, (controlador de domini, no crec que siga necessari, però es pot posar també, si volem controlar els recursos que tinga accés cada membre)

Es podia discutir si interessa un sistema de comunicacions intern, per poder comunicar mitjançant so, vídeo o xat entre el control i l'escenari. o volen passar informació en temps real als actors en l'escenari des de control. hi ha diverses opcions lliures, per exemple \href{https://meet.jit.si/}{Jitsi meet}, hauria de mirar-ho, ho he de provar varie vegades, i no he pogut fer-lo funcionar, es cosa d'insistir. Aço li donaria un poc mes de faena al servidor, però com no serien molts interlocutors, no crec que requerisca massa potencia.

Es podria posar també un sistema de \href{https://nextcloud.com/}{cloud} intern, que ja integra les videoconferències i moltes més coses per al treball en equip, però supose que amb un equip reduït, igual no val la pena, si soles es local, i la gent treballa des de casa la major part del temps. Es pot contractar un conte en línia, però ja s'està gastant el drive. Es pot fer una demo, i si interessa es deixa.

Sistema de correu intern, no li veig massa sentit, però es pot posar.

Al servidor de web, es pot instal·lar una còpia de WordPress, per si es vol experimentar com es veuen les coses abans de publicar-les. Es pot configurar perquè cada usuari de la sala tinga el seu directori per penjar les seues fulles.

Crec que seria interessant, per servir presentacions de les obres que es van a representar, si al final s'instal·la wifi en la sala, que tinguen una primera fulla al connectar, on es parle de l'obra que van a veure, de les pròximes que es representaran una fulla, o on es puga interactuar amb ells, passant qüestionaris, convidant-los a fer comentaris \ldots{} ( açò es intern i soles es pot realitzar amb la gent que es connecte al wifi)

En principi l'ordinador per fer el streaming, ha de tindre una targeta de vídeo mitjanament potent, no crec que siga el cas del servidor, aquest no requereix ni pantalla una vegada instal·lat, es gastaria algun altre disponible en la sala, en cas necessari, ho podíem fer també en aquest, però va augmentant el nivell de requisits mínims de potència.

\texttt{Si\ se\ m\textquotesingle{}ocorre\ alguna\ cosa\ més\ la\ posaré,\ o\ ja\ em\ digueu.}

\hypertarget{servidor-de-disc}{%
\subsection{Servidor de disc}\label{servidor-de-disc}}

Si es volen guardar les còpies de les càmeres, i la idea es no fer un streaming pur, més bé editar a posteriori les imatges, o tindre una còpia dels assajos per veure en què es pot millorar, es requeriria espai per fer aquestes còpies. Hi ha dues possibilitats.

\begin{itemize}
\item
  \href{https://en.wikipedia.org/wiki/Network-attached_storage}{NAS}, sistema de disc en xarxa, La millor opció.
\item
  Que la torre que gastem de servidor , faça de servidor de disc, on es pot implementar configurar RAID en diferents tipus de redundàncies, segons el nombre de discs durs que tinguem, i si ens interessa més la velocitat o la redundància.
\end{itemize}

\hypertarget{cuxe0meres-ip}{%
\subsection{Càmeres ip}\label{cuxe0meres-ip}}

Hi ha dos tipus de càmeres ip que ens vindrien bé, unes són del tipus \href{https://ca.wikipedia.org/wiki/C\%C3\%A0mera_PTZ}{PTZ} es poden controlar remotament, i altres que són fixes, més barates. Es podria fer una mix dels dos tipus, la frontal a l'escenari, podria fer fixa, i en algun lateral posar una amb moviment.

El projecte es pot implantar per fases, primer provar en una, i si dona resultat anar ampliant, segons necessitats.

Per les característiques dels espectacles, es requeriria càmeres amb bona lluminositat, bones lents, açò augmentara el pressupost, pero segons les capacitats es pot anar fent, i millorar si es considera que val la pena.

Les càmeres i el muntatge està pensat per poder ser fàcilment desmuntable i transportat en cas de tindre la necessitat de realitzar l'espectacle extern a la sala. O reutilitzar el material obsolet per aquest fi.

\texttt{Llista\ provisional\ de\ recerca\ de\ càmeres\ i\ marques.}

\begin{itemize}
\item
  \href{https://www.amazon.es/dp/B086X637W2?ref_=as_li_ss_tl\&language=en_US\&linkCode=gs2\&linkId=a19d43bd6c875dd9cde44b1b4f5a3776\&tag=getlockers0f8-21\&th=1}{camera en amazon zowietek}
\item
  \href{https://www.axis.com/en-us/products/axis-v59-series}{Axis}
\item
  \href{https://getlockers.com/best-ptz-camera-for-live-streaming/}{Getlockers}
\item
  \href{https://ptzoptics.com/sdi/}{Ptzoptics}
\end{itemize}

\hypertarget{el-compte-del-streaming}{%
\subsection{El compte del streaming}\label{el-compte-del-streaming}}

És traure un compte en alguna de les plataformes que existeixen, ja teniu un compte en youtube, es pot gastar eixa o traure en \href{https://www.twitch.tv/}{twitch}, a veure quina deixa configurar més la pàgina d'inici, es pot crear una nova on enllaçar els continguts, o integrar-los en les fulles de l'empresa.

\hypertarget{muntatge}{%
\section{Muntatge}\label{muntatge}}

Faré proves de tot açò i penjaré el resultat de com es fa.

\hypertarget{installaciuxf3-de-wifi-per-al-pati-de-butaques}{%
\chapter{Instal·lació de wifi per al pati de butaques}\label{installaciuxf3-de-wifi-per-al-pati-de-butaques}}

\hypertarget{requeriments-per-a-la-seua-implantaciuxf3}{%
\section{Requeriments per a la seua implantació}\label{requeriments-per-a-la-seua-implantaciuxf3}}

Uns repetidors de wifi, he quedat en un amic que controla del tema per a mirar-ho. S'haguera de pensar si realment es vol instal·lar per tindre cobertura per als 92 espectadors, o es pensa que no tots es connectaran, i començar per un sistema bàsic, ampliable a mesura que augmenten les necessitats.

\hypertarget{potuxe8ncia-de-la-connexiuxf3-dinternet-per-donar-servei}{%
\subsection{Potència de la connexió d'internet per donar servei}\label{potuxe8ncia-de-la-connexiuxf3-dinternet-per-donar-servei}}

Es requereix que l'amplada de banda de la connexió d'internet puga donar servei almenys a 92 espectadors, que es l'afore del pati de butaques.

S'ha de limitar l'amplada de banda que oferim als espectadors en cas de realitzar el streaming al mateix temps, açò ho realitzem assegurant l'amplada de banda en la xarxa del circuit d'imatge. En el switch.

\hypertarget{localitzaciuxf3-dels-dispositius-de-retransmissiuxf3}{%
\subsection{localització dels dispositius de retransmissió}\label{localitzaciuxf3-dels-dispositius-de-retransmissiuxf3}}

Hem d'estudiar el número i posicions en la sala perquè la retransmissió siga òptima.

\hypertarget{muntatge-1}{%
\section{Muntatge}\label{muntatge-1}}

\texttt{Aci\ descriure\ com\ es\ realitza\ l\textquotesingle{}instal·lacio\ i\ els\ components\ necessaris.}

\hypertarget{implantaciuxf3-dentorn-git}{%
\chapter{Implantació d'entorn Git}\label{implantaciuxf3-dentorn-git}}

Explorar la forma d'integrar el concepte git en les arts escèniques. Explorar els avantatges del seguiment de versions, i el procés col·laboratiu per a l'elaboració de publicacions o la creació d'obres de teatre,

\hypertarget{implantaciuxf3-del-sistema-collaboratiu-en-el-desenvolupament-dobres-de-teatre.}{%
\section{Implantació del sistema col·laboratiu en el desenvolupament d'obres de teatre.}\label{implantaciuxf3-del-sistema-collaboratiu-en-el-desenvolupament-dobres-de-teatre.}}

\begin{itemize}
\tightlist
\item
  Es pot realitzar mitjançant la creació d'un compte en una de les plataformes Git que hi ha en internet,
\item
  Configurant un servidor git en el mateix centre de treball

  \begin{itemize}
  \tightlist
  \item
    Git i un servidor web, aquesta opció es per a gastar l'entorn git en consola, no es recomana per a l'empresa.
  \item
    Un entorn web, he fet proves en GitLab i Gitea, i el final crec que la millor opció és la de Gitea pel reduït gaste de recursos que fa i la facilitat d'ús.
  \item
    Es pot mirar d'enllaçar el repositori git en el conte de drive que té l'empresa, per fer un repositori privat per als membres.
  \end{itemize}
\end{itemize}

\hypertarget{beneficis-de-la-seua-implantaciuxf3}{%
\section{Beneficis de la seua implantació}\label{beneficis-de-la-seua-implantaciuxf3}}

Els beneficis són els mateixos que per a la creació de projectes informàtics, però que son poc utilitzat en el món de les lletres. \emph{Possibilitat de crear obres col·laboratives.}

\begin{itemize}
\tightlist
\item
  El guardat de les diferents versions segons va desenvolupant-se el projecte, i poder tornar en el temps si alguna cosa no va bé.
\item
  Bifurcar camins d'exploració en la creació literària.
\end{itemize}

\hypertarget{mode-de-funcionament}{%
\section{Mode de funcionament}\label{mode-de-funcionament}}

\hypertarget{breu-tutorial-per-al-funcionament}{%
\subsection{Breu tutorial per al funcionament}\label{breu-tutorial-per-al-funcionament}}

\hypertarget{installaciuxf3-dun-repositori-en-el-centre}{%
\section{Instal·lació d'un repositori en el centre}\label{installaciuxf3-dun-repositori-en-el-centre}}

Procés d'instal·lació d'un repositori de git

\begin{itemize}
\item
  Git i Apache \texttt{buscar\ la\ fulla\ on\ ho\ explica}
\item
  Instal·lació de \href{https://docs.gitea.io/en-us/}{gitea} , o bé directament, o per una imatge Docker
\item
  Servir el \href{https://www.permikkelsen.dk/how-to-host-your-git-repository-on-onedrive.html}{git} en un espai onedrive, pareix que és per consola, mirar algun client de git desktop que facilita les coses.
\end{itemize}

\hypertarget{possible-implantaciuxf3-en-el-repositori-en-internet}{%
\section{Possible implantació en el repositori en internet}\label{possible-implantaciuxf3-en-el-repositori-en-internet}}

El més fàcil seria traure un compte en una de les plataformes que hi ha en internet, el problema és el recel que tinga l'empresa pel tema de privacitat. Aquestes plataformes tenen l'opció de crear repositoris privats, on soles es pot connectar al que convideu.

\hypertarget{explorar-obsidian}{%
\section{\texorpdfstring{Explorar \href{https://obsidian.md/}{Obsidian}}{Explorar Obsidian}}\label{explorar-obsidian}}

Obsidian és un editor de notes en funcions de seguiment git, per veure diff i altres funcions. Prou interessant, o com a alternativa \href{https://www.zettlr.com/}{Zettlr}, dona error en posar diccionaris.\\
I implantació d'un servidor de \href{https://github.com/Erikvl87/docker-languagetool}{Languagetool}

\hypertarget{explorar-wikimedia-per-a-la-realitzaciuxf3-descriptura-collaborativa}{%
\section{Explorar Wikimedia per a la realització d'escriptura col·laborativa}\label{explorar-wikimedia-per-a-la-realitzaciuxf3-descriptura-collaborativa}}

L'escriptura col·laborativa mitjançant ferramentes de xarxa, com la Wikipedia, on és possible discutir els convits proposats, obrir discussions sobre el text analitzat. Es pot explorar el concepte de \href{https://es.wikipedia.org/wiki/Narrativa_hipertextual}{Narrativa hipertextual} fent servir aquesta ferramenta.

\hypertarget{migraciuxf3-de-les-fulles-web-a-hugo}{%
\chapter{Migració de les fulles web a Hugo}\label{migraciuxf3-de-les-fulles-web-a-hugo}}

\texttt{En\ principi\ descartat}

\hypertarget{avantatges-de-la-migraciuxf3}{%
\section{Avantatges de la migració}\label{avantatges-de-la-migraciuxf3}}

\emph{Hugo és un framework per a construir fulles estàtiques}

La fulla web de l'empresa està implementada en WordPress,Hugo és una alternativa per la realització del projecte d'aquesta fulla, on el seu manteniment i implementació és molt més senzilla. Hugo és un framework on una vegada configurat el tema de la fulla, els autors de les publicacions, soles han de penjar un document de text, en format Markdown perquè aquests nou posts es publique, l'organització en carpetes i mitjançant tags ens facilita molt l'administració del lloc, a més de la seua seguretat, ja que soles compartim fulles estètiques, sense fer ús de bases de dades, reduint considerablement el seu manteniment.

\hypertarget{mode-dimplementar-la-migraciuxf3}{%
\section{Mode d'implementar la migració}\label{mode-dimplementar-la-migraciuxf3}}

Una vegada que l'empresa passe la còpia de seguretat de les fulles WordPress, hi ha una forma de passar els posts a Markdown.

Ordenant en diferents carpetes, per contingut temàtic i autors, on cada autor o secció te la seua carpeta on colgar els nous posts, en fitxer de text \emph{md} i les imatges que vulga que apareguen.

\hypertarget{conversiuxf3-dels-posts-de-wordpress-a-md}{%
\subsection{Conversió dels posts de WordPress a md}\label{conversiuxf3-dels-posts-de-wordpress-a-md}}

Possibles solucions per al procés

\begin{itemize}
\tightlist
\item
  \href{https://kevq.uk/how-to-convert-wordpress-to-markdown/}{How To Convert WordPress To Markdown}
\item
  \href{https://swizec.com/blog/how-to-export-a-large-wordpress-site-to-markdown/}{How to export a large Wordpress site to Markdown}
\item
  \href{https://prefetch.net/blog/2017/11/24/exporting-wordpress-posts-to-markdown/}{Exporting Wordpress Posts To Markdown}
\end{itemize}

\emph{Quan estiguem en el cas estudiarem en més profunditat la manera de fer-ho}

Estic mirant per a fer una xicoteta demostració de les característiques d'Hugo, per fer una posada en comú de com funciona, i si és factible la seua migració.

Una altra possibilitat de la seua implantació, és si es realitza la instal·lació d'una zona wifi en el pati de butaques, a l'hora de donar els paràmetres de la xarxa (nom de xarxa i contrasenya), que es faria en un codi \href{https://www.qr-code-generator.com/solutions/wifi-qr-code/}{QR} \emph{(alguna cosa pareguda açò, ja ho desenvoluparé un poc mes en l'apartat wifi)}, es pot passar una direcció d'entrada automàtica a una fulla web, que estaria en el servidor de la xarxa local, on es pot passar informació de l'obra que és representar, o realitzar interacció amb el públic, en temps real.

\hypertarget{possibles-interaccions-amb-el-puxfablic}{%
\subsubsection{Possibles interaccions amb el públic}\label{possibles-interaccions-amb-el-puxfablic}}

Les fulles del servidor intern per al públic, recomane que es facen amb Hugo, ho deixaré preparat perquè siga molt fàcil la creació de fulles noves. On soles s'ha de colgar un fitxer de text en un directori determinat\ldots{} bo, i compilar la fulla, però ja deixaré un script que ho faça, miraré la forma de fer-ho soles, integració continuada crec que es diu. Es pot fer en WordPress si vos és més fàcil, és parlar-ho i provar.

\begin{itemize}
\item
  Preàmbul de l'obra que es va a representar, perquè el públic s'informe o posar en context de què van a veure, uns minuts abans que comence.
\item
  Es pot dur a terme una web app de xat per a conferències, el públic puga colgar allí les seues preguntes, i des de la taula, poder seleccionar les més interessants, o la persona a la qual es donara veu.
\item
  Fer obres interactives amb el públic, on aquest puga fer votacions, mitjançant el seu mòbil, i d'aquesta forma poder canviar el desenllaç de l'obra en temps real.
\end{itemize}

En definitiva, si es configura una xarxa en la sala, poder traure tot el rendiment possible, ja que en l'actualitat tot el món té un mòbil damunt.

\hypertarget{organitzaciuxf3-dels-diferents-apartats-de-les-fulles}{%
\section{Organització dels diferents apartats de les fulles}\label{organitzaciuxf3-dels-diferents-apartats-de-les-fulles}}

\hypertarget{alternativa-merege-wordpress}{%
\section{Alternativa, merege WordPress}\label{alternativa-merege-wordpress}}

En cas contrari, es miraria la forma de combinar els 5 WordPress en un sol, fer \href{https://wpmudev.com/blog/merging-wordpress-sites/}{merge wordpress} \texttt{mirar\ que\ més\ alternatives\ hi\ ha.}

\hypertarget{implantaciuxf3-de-projectes-exelearning}{%
\chapter{Implantació de projectes Exelearning}\label{implantaciuxf3-de-projectes-exelearning}}

\href{https://exelearning.net/ca}{Exelearning} és una eina d'autor de codi obert per ajudar als docents en la creació i publicació de continguts web.

\hypertarget{proposta-dimplantar-donar-cursos-en-la-sala-o-patrocinats-per-la-mateixa}{%
\section{Proposta d'implantar donar cursos en la sala o patrocinats per la mateixa}\label{proposta-dimplantar-donar-cursos-en-la-sala-o-patrocinats-per-la-mateixa}}

Una de les activitats de l'empresa és en l'entorn educatiu, de la realització d'obres de teatre en instituts i col·legis. Una bona forma d'introduir la digitalització en les arts escèniques, seria preparar cursos introductors a l'obra que es vol representar. El professorat els poguera impartir abans de la seua representació, i anar preparant als alumnes del que van a veure, o fer cursos o jocs interactius post espectacle per reforçar el missatge que es vol transmetre.

\hypertarget{funcionament-dexelearning}{%
\section{Funcionament d'exelearning}\label{funcionament-dexelearning}}

Anem a \href{https://exelearning.net/ca/descarregues/}{descàrregues} per obtenir el programa per a la plataforma on vulguem construir el projecte. En el següent enllaç \href{https://descargas.intef.es/cedec/exe_learning/Manuales/manual_exe26/}{\textbf{Documentació}}, tenim una explicació de com funciona i les possibilitats d'exportar el projecte siga a la nostra fulla o a espais gratuïts d'allotjament de contingut educatiu que després podem enllaçar.

\hypertarget{llocs-on-podem-penjar-el-material-educatiu}{%
\subsection{Llocs on podem penjar el material educatiu}\label{llocs-on-podem-penjar-el-material-educatiu}}

\begin{itemize}
\tightlist
\item
  \href{https://public.bscw.de/pub/}{BSCW (Basic Support for Cooperative Work)}
\item
  \href{https://graasp.eu/}{Graasp} Pertany a la comunitat europea i la seua finalitat és crear i allotjar material educatiu.
\item
  \href{http://vishub.org/}{Vish} Xarxa social educativa per la creació de recursos educatius.
\end{itemize}

\hypertarget{possible-implantaciuxf3-duna-plataforma-moodle}{%
\section{Possible implantació d'una plataforma Moodle}\label{possible-implantaciuxf3-duna-plataforma-moodle}}

Si es volguera anar un poc més lluny, es podria implantar una plataforma Moodle, per donar cursos relacionats en les arts escèniques, on es colgaria material, hi ha diversos allotjaments on podem penjar el nostre Moodle a preus raonables, per fer un curs que ens interesse en un moment donat, mirar \href{https://www.hostingadvice.com/how-to/best-moodle-hosting/}{Best Moodle Allotjament Services}

\hypertarget{desenvolupament-appweb}{%
\chapter{Desenvolupament Appweb}\label{desenvolupament-appweb}}

\emph{L'empresa vol personalitzar informes per a la seua comptabilitat i la demanda de subvencions, on es reflectiran el nombre d'obres, espectadors, gasto, i fer estadístiques de l'assistència de públic i comparar en Google analític}.

\hypertarget{desenvolupament-duna-app-de-gestiuxf3-dinformes-de-la-sala}{%
\section{Desenvolupament d'una app de gestió d'informes de la sala}\label{desenvolupament-duna-app-de-gestiuxf3-dinformes-de-la-sala}}

\emph{Es proposa la realització d'una appweb que realitze aquestes tasque.}

En principi, seria mitjançant una base de dades, on introduir els valors que recopile l'empesa, i un entorn web on es puguen fer consultes i redactar informes.

\begin{quote}
Pendent de resoldre, la conveniència o no de la utilització d'una base de dades, una SQLite o un fitxer json, ja que el nombre de dades no és massa elevat.\\
Investigar com es fan aquestes coses, ja que jo soc de xarxes, no de programació.
\end{quote}

L'app es colgaria en el servidor de l'empresa, perquè es puguen fer consultes des d'allí. El que implica la configuració d'un servidor en la mateixa sala, on de pas posarem un servidor de dns, un de fulles web si es vol realitzar interaccions amb el públic, optatiu un de correu, i es podria imantar també un servidor git, per fer les còpies de seguretat i seguiment de versions, dels projectes com la revista que publiquen o obres de teatre, en el cas que les vulguen crear de forma col·laborativa.

Després de provar diverses alternatives, com \href{https://about.gitlab.com/}{gitlab}, \href{https://gogs.io/}{Gogs}, la millor alternativa pareix ser \href{https://gitea.io/en-us/}{Gitea}, és molt lleuger, es pot instal·lar amb una base de dades SQLite, i l'entorn web que proporciona és molt paregut al de Github, i les exigències de l'empresa, no són les d'un entorn de programació.

La millor opció per aquest cas és traure un compte de GitHub de l'empesa, on es guarden els projectes.

\hypertarget{alternatives}{%
\subsection{Alternatives}\label{alternatives}}

\begin{itemize}
\tightlist
\item
  El projecte es pot realitzar en node.js, express i vue.js
\item
  \href{https://www.r-project.org/}{R} i \href{https://shiny.rstudio.com/}{shiny}, que pareix més fàcil, el que implica la instal·lació de R en el servidor, perquè ens servisca l'aplicació.
\end{itemize}

\hypertarget{tasques-a-realitzar}{%
\section{Tasques a realitzar}\label{tasques-a-realitzar}}

\begin{itemize}
\tightlist
\item
  S'ha d'investigar la programació de plantilles de \href{http://www.latextemplates.com/}{Latex}, per a formatjar els documents d'eixida, en el logo de l'empresa, i un format propi de redaccions d'informes.
\item
  El funcionament de \href{https://pandoc.org/}{Pandoc}, per a poder traure els diferents formats que necessitem, pdf, docx \ldots{}
\item
  Crear un xicotet doc del funcionament de markdown per a la gent de l'empresa.
\end{itemize}

\hypertarget{remodelat-de-la-fulla-web-de-espacio-inestable}{%
\chapter{\texorpdfstring{Remodelat de la fulla web de \emph{Espacio Inestable}}{Remodelat de la fulla web de Espacio Inestable}}\label{remodelat-de-la-fulla-web-de-espacio-inestable}}

\hypertarget{reformulaciuxf3-de-les-estratuxe8gies-de-venda-dentrades}{%
\section{Reformulació de les estratègies de venda d'entrades}\label{reformulaciuxf3-de-les-estratuxe8gies-de-venda-dentrades}}

L'estratègia de la venda d'entrades, és gestionada per una empresa de venda d'actes, es planteja la possibilitat de migrar a una altra plataforma o autogestió d'aquestes.

\hypertarget{installaciuxf3-del-plugin-de-venda-dentrades-i-tema-per-a-la-fulla-web}{%
\section{Instal·lació del plugin de venda d'entrades i tema per a la fulla web}\label{installaciuxf3-del-plugin-de-venda-dentrades-i-tema-per-a-la-fulla-web}}

Es recomana d'implementació d'una solució com la que és té en aquesta fulla, on podem veure una demostració de les seues funcions.\href{https://tickera.com/demos/theater-demo/}{ticketera}

En aquest plugin per a WordPress, tenim la possibilitat de configurar les nostres entrades, afegint imatge de l'obra, inclusió de codis Qr (per al wifi), implantar la compra d'entrades, amb selecció de la cadira en el pati de butaques.

S'hauria de contractar els serveis de \href{https://stripe.com/en-es}{stripe}, la passerella de cobrament. El \href{https://stripe.com/en-es/pricing}{preu} és 1,4\% +0,25\,€, la configuració en WordPress és molt fàcil, i el tema i el plugin ja estan preparats.

Una altra part interessant d'aquest plugin, és la possibilitat de validació d'entrades mitjançant un codi Qr, que funciona en una aplicació de movil, actualitzant la base de dades d'assistència real a l'acte.

\hypertarget{tasques-a-fer}{%
\section{Tasques a fer}\label{tasques-a-fer}}

Configurar el tema de la fula web perquè tinga l'aspecte desitjat, es podria comprar també un tema que ofereixen en la mateixa ticketera dissenyat expressament per a \href{https://tickera.com/demos/theater-demo/}{teatres} i actes.

\hypertarget{configuraciuxf3-del-servidor}{%
\chapter{Configuració del servidor}\label{configuraciuxf3-del-servidor}}

Descriurem com seria la posada en funcionament del servidor.

Per dur a terme aquest projecte anem a fer una simulació en una màquina virtual en un Ubuntu server que descarregarem \href{https://ubuntu.com/download/server}{d'Ubuntu server}. I simularem que tenim 3 discs durs per a fer un RAID.

\href{https://tonyfernandeztech.wordpress.com/2021/04/27/installing-ubuntu-20-04-with-software-raid-1/}{Ubuntu en raid}

Anem a fer la instal·lació en format \href{https://en.wikipedia.org/wiki/Standard_RAID_levels}{RAID} 5, farien falta 3 discos, però ens assegurem recuperació contra fallades, es pot fer també amb raid 0 on augmenta la velocitat, però no tenim recuperació en fallides d'un disc, es perd tot.

Hem de decidir si volem el tipus 0, on es té tot l'espai dels discs durs, i augmentem la velocitat de funcionament, o fem un 5 on l'espai d'un dels discs es perd per a fer còpia de paritat, i no és tan ràpid, pel tipus d'empresa, la velocitat punta del 0 no és necessària, i jo primaria la seguretat.

\emph{Recomane el tipus 5.}

Hem de comprar 3 hd sata depen de la mida valdrà un preu, per exemple \href{https://www.amazon.es/Blue-256MB-3-5IN-SATA-Sint/dp/B08VH8R94B/ref=sr_1_4?hvadid=79852063937015\&hvbmt=be\&hvdev=c\&hvqmt=e\&keywords=hd+2tb+interno\&qid=1650128602\&refinements=p_n_size_browse-bin\%3A10858611031\&rnid=949713031\&s=computers\&sr=1-4}{WD Blue 2TB SATA} està en Amazon per 44 euros, per 3 discs, 132 euros, i tindríem una capacitat d'emmagatzemament de 4 TB en recuperació si falla un. Si no, es compren 2 per 88 i no fem un tipus 0.

\begin{rmdnote}{No es el nostre cas}
Exemple de configuració de \href{https://tonyfernandeztech.wordpress.com/2021/04/27/installing-ubuntu-20-04-with-software-raid-1/}{RAID 1} en aquest tipus es fa una còpia espill dels discs. es perd el 50\% de l'espai total per fer la còpia, no crec que interesse. \emph{No es el cas que ens interessa.}

\end{rmdnote}

\begin{rmdinfo}{Muntar un RAID 5}
Aci tenim com muntar un \href{https://www.tutorialspoint.com/how-to-create-a-raid-5-storage-array-with-mdadm-on-ubuntu-16-04}{RAID 5 en Ubuntu}, ja en funcionament. Nosaltres començarem des de zero, fent una instal·lació neta.

\end{rmdinfo}

Mira açò, igual es millor fer un vlm i posar raid després, o formatar en \href{https://help.ubuntu.com/community/btrfs}{btrsf} \href{https://hetmanrecovery.com/recovery_news/how-to-create-software-raid-5-with-lvm.htm}{LVM+RAID}

Possibilitat de no fer un RAID , fer un lvm en diversos discs i posar sistema de fitxer btrfs en compressió, encara que si el que es guarda es majoritàriament vídeos, no te massa sentit, es perd un poc de rendiment, i si el server no es massa potent, no valdria la pena, te millores en la recuperació d'erros, i em pareix millor per a un sistema en poc manteniment per al futur.

\emph{El que cal pensar ara} és, sistema en home en espai per a usuaris, o muntar en mnt/espai perquè es guarden els vídeos.

\begin{itemize}
\tightlist
\item
  Per una part, tindre carpetes d'usuaris, estaria be, si es posa al final servidor web, i fer que cada usuari servisca des del seu directori públic, o montar el directori per xarxa, per què guarden allí fitxers de recuperació.
\item
  El més segur es que no ho gasten mai. L'opció de fer sol servidor de disc dur, i posar allí una carpeta perquè colguen el que es vol servir per web interna, igual seria millor opció.
\end{itemize}

\begin{rmdnote}{No és el nostre cas}
Més que res, és pe a l'hora de fer les particions del disc dur, li donem més a home, o posem el pes de l'espai a la partició d'espai. Clarament, les he de separar, ja que no vull que per una d'aquestes es quede cense espai i que el servidor comence anar malament o directament no funcione per falta d'espai, i més si es guarden allí les captures de reproducció en línia, que per si una d'aquestes, es queda encés, el pot deixar cense disc en poc de temps.

\end{rmdnote}

Resoldre el problema dels permisos, que grups crear, que usuaris i que pot fer que.

\hypertarget{altres-coses-a-fer}{%
\section{Altres coses a fer}\label{altres-coses-a-fer}}

\begin{itemize}
\item
  Possible instal·lació d'un NAS
\item
  Fer les funcions de NAS en el mateix servidor, fer RAID, depenent del pressupost, fer un de tipus 0, 1 o 5.
\item
  Configurar un DNS en el servidor.
\item
  Configurar un DHCP, de moment ho descartem, si anara a més el wifi en la sala, i es vulga fer més interaccions amb el públic, seria per a estudiar, soles per al número d'ordinador del centre, amb IP fixes, va que el mates.
\item
  Servidor d'impressió, Configurar el server com a servidor d'impressió.
\item
  Configurar un servidor web, per als serveis que es poden posar.
\item
  La web de, es poden posar WordPress per a cada usuari, perquè facen les seues proves, o intentar convèncer de les bondats d'Hugo o Jenkyll, estudiar la seua configuració, creació d'un tema adequat per a la sala, i explicar la forma que el puguen gastar.
\end{itemize}

\hypertarget{notes}{%
\chapter{Notes}\label{notes}}

Recordatoris de com es fan les coses, després el esborrare

\hypertarget{recordatoris}{%
\section{Recordatoris}\label{recordatoris}}

\textbf{Pera que els gif no vagen al pdf}

Falta saber com es canvia per un png

```\{r sheets-option-drag, fig.pos=`h', fig.cap=``Prova de gif''\}\\
if(knitr::is\_html\_output(excludes=``markdown'')) knitr::include\_graphics(``imatges/provacomp.gif'')\\
```

\textbf{Per a comprimir els gif}

\begin{Shaded}
\begin{Highlighting}[]
\ExtensionTok{$}\NormalTok{ gifsicle }\AttributeTok{{-}O3} \AttributeTok{{-}{-}colors}\OperatorTok{=}\NormalTok{64 }\AttributeTok{{-}{-}use{-}col}\OperatorTok{=}\NormalTok{web }\AttributeTok{{-}{-}lossy}\OperatorTok{=}\NormalTok{100 prova.gif }\AttributeTok{{-}o}\NormalTok{ comgif.gif  }
\ExtensionTok{o}\NormalTok{ no tan bestia  }
\ExtensionTok{$}\NormalTok{ gifsicle }\AttributeTok{{-}O3} \AttributeTok{{-}{-}lossy}\OperatorTok{=}\NormalTok{80 prova.gif }\AttributeTok{{-}o}\NormalTok{ provacomp1.gif  }
\end{Highlighting}
\end{Shaded}

\begin{rmdinfo}{}
::: \{.rmdinfo .centre data-latex=``\{\}''\}\\
la la la\\

\end{rmdinfo}

:::

\begin{rmdcuidao}{Ves en conter}
\{.rmdcuidao data-latex=``\{ves en conter\}''\}

\end{rmdcuidao}

\begin{rmdwarn}{Perill}
\{.rmdwarn data-latex=``\{Perill\}''

\end{rmdwarn}

\begin{rmdtip}{Tip}
\{.rmdtip data-latex=``\{Tip\}''\}

\end{rmdtip}

\begin{rmdnote}{Nota}
\includegraphics{imatges/note.png} \textbf{Aço es una nota} \{.rmdnote data-latex=``\{Nota\}''\}\\
ves en cuidaoooo no se si açò tancara o continua fins el mas enllà\\
ves en cuidaoooo no se si açò tancara o continua fins el mas enllà

\end{rmdnote}

\begin{rmdinfo}{}
\textbf{Perquè no isca en el pdf}

```\{asis, echo=!knitr::is\_latex\_output()\}\\
::: \{.rmdcuidao\}\\
ves en cuidaoooo no se si açò tancara o continua fins el mas enllà :::\\
```

\end{rmdinfo}

  \bibliography{book.bib,packages.bib}

\end{document}
